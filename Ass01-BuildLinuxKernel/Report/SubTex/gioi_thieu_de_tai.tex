\section{Trạm dừng không được phân bố đồng đều dẫn đến khó khăn trong việc bắt xe bus.}

Giải pháp:

\begin{itemize}
	\item Đặt trạm ở vị trí thuận lợi.
	\item Thu hẹp khoảng cách giữa các trạm.
	\item Các trạm cần có bản đồ xe bus.
\end{itemize}




\section{Tình trạng kẹt xe ở trạm dừng gây khó khăn cho việc bắt xe, quá tải hành khách trên xe.}

Giải pháp:

\begin{itemize}
	\item Tăng cường số xe chạy trên các tuyến.
	\item Mở rộng qui mô trạm dừng để thuận tiện cho xe ra vào trạm, hành khách đón xe dễ dàng.
	\item Khi nhiều xe vào trạm, tài xế xe sau bị khuất tầm nhìn bởi xe trước, không biết hành khách đang chờ xe, khiến hành khách lỡ chuyến.
	\item Xây dựng làn đường riêng cho xe bus.
	\item Bố trí hợp lí thời gian xuất bến, cập bến các tuyến xe một cách hợp lí.
	\item Giải tán các điểm tụ tập buôn bán tại trạm dừng.
\end{itemize}

\section{Nhiều trạm không có chỗ ngồi, mái che, nhà vệ sinh công cộng gây khó khăn cho hành khách.}

Giải pháp:

\begin{itemize}
	\item Tăng không gian trạm dừng.
	\item Xây dựng nhà chờ có điều hòa, tránh mưa nắng tốt kết hợp với nhà vệ sinh công cộng.
\end{itemize}

\section{Vấn đề an ninh tại trạm dừng.}

Giải pháp:

\begin{itemize}
	\item Lập các đội bảo vệ gần trạm dừng.
	\item Lắp đặt camera.
\end{itemize}



\section{Hành khách không biết chính xác thời điểm xe tới.}

Giải pháp:

\begin{itemize}
	\item Lắp đặt bản đồ xe bus điện tử thông minh tại các trạm.
	\item Phát triển ứng dụng bản đồ thông minh trên các thiết bị di động cho người dùng.
	\item Lắp đặt biển báo dự kiến thời điểm xe vào trạm
\end{itemize}