\section{Prepare Linux Kenel}

\subsection{QUESTION: Why we need to install kernel-package?}

kernel-package is utility for building Linux kernel related Debian packages. The package automate the routine steps required to compile and install a custom kernel.

\noindent For more advantages, see \url{http://man.he.net/man5/kernel-package}

\subsection{QUESTION: Why we have to use another kernel source from the server such as http://www.kernel.org, can we compile the original kernel (the local kernel on the running OS) directly?}



\section{System Call - procsched}

\subsection{QUESTION: What is the meaning of other parts, i.e. i386, procsched, and sys procsched?}

Each system call is declared in one row with following information: number, ABI, name, entry point and compat entry point.


\subsection{QUESTION: What is the meaning of each line added in file include/linux/syscalls.h?}


\section{Compiling Linux Kernel}

\subsection{QUESTION: What is the meaning of these two stages, namely “make” and “make modules”?}


\subsection{QUESTION: Why this program could indicate whether our system works or not?}



\section{Wrapper}

\subsection{QUESTION: Why we have to re-define proc segs struct while we have already defined it inside the kernel?}



\subsection{QUESTION: Why root privilege (e.g. adding sudo before the cp command) is required to copy the header file to /usr/include?}


\subsection{QUESTION: Why we must put -share and -fpic option into gcc command?}

